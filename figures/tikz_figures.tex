%% =============================================================================
%% FIGURAS TIKZ PARA PAPER 0A
%% =============================================================================
%% Incluir en el preámbulo del paper:
%% \usepackage{tikz}
%% \usetikzlibrary{decorations.pathmorphing,patterns,arrows.meta,shapes,calc}
%% =============================================================================


%% =============================================================================
%% FIGURA 1: TWISTOR SPACE DECOMPOSITION
%% =============================================================================
%% Ubicación sugerida: Después de Sección 4.4

\begin{figure}[htbp]
\centering
\begin{tikzpicture}[scale=1.3]
  % Twistor space as ellipse
  \fill[blue!5] (0,0) ellipse (3 and 2);
  \draw[thick] (0,0) ellipse (3 and 2);
  
  % Division line (PT_R)
  \draw[thick, dashed, red!70!black] (-3,0) -- (3,0);
  
  % Labels for regions
  \node at (0,1) {\large $\PT^+$};
  \node at (0,-1) {\large $\PT^-$};
  \node[red!70!black] at (3.4,0) {$\PT_\RR$};
  
  % Twistor line CP1_x
  \draw[thick, blue!70!black, line width=1.5pt] (-2,-1.5) -- (1.5,1.8);
  \node[blue!70!black] at (1.9,2) {$\CP^1_x$};
  
  % Hemisphere H_x (portion of line in PT+)
  \draw[very thick, green!50!black, line width=2pt] (-0.5,0) -- (1.5,1.8);
  \node[green!50!black] at (-0.2,0.8) {$H_x$};
  
  % Intersection point with PT_R
  \fill[red!70!black] (-0.5,0) circle (3pt);
  
  % Arrow to spacetime
  \draw[-{Stealth[length=3mm]}, thick] (4,0) -- (6,0);
  \node at (5,0.4) {incidence};
  
  % Spacetime point
  \fill[black] (7,0) circle (4pt);
  \node at (7,-0.4) {$x \in M$};
  
  % Annotation
  \node[align=center, font=\small] at (0,-2.7) {
    Positive frequency $\Leftrightarrow$ holomorphic on $H_x$
  };
\end{tikzpicture}
\caption{Twistor space decomposition and the incidence relation. The projective twistor space $\PT \cong \CP^3$ is divided by the real structure into regions $\PT^+$, $\PT^-$, and $\PT_\RR$. The twistor line $\CP^1_x$ corresponding to spacetime point $x$ intersects $\PT^+$ in the hemisphere $H_x$. Cohomology classes that extend holomorphically to $H_x$ correspond to positive-frequency fields at $x$.}
\label{fig:twistor-decomposition}
\end{figure}


%% =============================================================================
%% FIGURA 2: CONCEPTUAL FLOW DIAGRAM
%% =============================================================================
%% Ubicación sugerida: Introducción o Sección 8

\begin{figure}[htbp]
\centering
\begin{tikzpicture}[
  node distance=1.8cm,
  box/.style={rectangle, draw, thick, minimum width=3.2cm, minimum height=0.9cm, align=center, font=\small},
  arrow/.style={-{Stealth[length=2.5mm]}, thick}
]
  % Nodes
  \node[box, fill=blue!10] (conf) {Conformal structure\\$[g]$};
  \node[box, fill=blue!10, right=of conf] (twistor) {Twistor space\\$\PT$};
  \node[box, fill=green!10, right=of twistor] (cohom) {Cohomology\\$H^1(\CP^1_x, \scrO)$};
  
  \node[box, fill=orange!10, below=of cohom] (holo) {Holomorphic\\content};
  \node[box, fill=red!10, below=of holo] (rho) {Conformal density\\$\rho(x)$};
  
  \node[box, fill=purple!10, left=of rho] (E) {Integrability\\$\mathcal{E}[\rho] = 0$};
  \node[box, fill=yellow!10, left=of E] (dyn) {Dynamics\\(emergent)};
  
  % Arrows
  \draw[arrow] (conf) -- (twistor);
  \draw[arrow] (twistor) -- (cohom);
  \draw[arrow] (cohom) -- (holo);
  \draw[arrow] (holo) -- (rho);
  \draw[arrow] (rho) -- (E);
  \draw[arrow] (E) -- (dyn);
  
  % Annotations
  \node[font=\scriptsize, above] at ($(conf)!0.5!(twistor)$) {correspondence};
  \node[font=\scriptsize, above] at ($(twistor)!0.5!(cohom)$) {Penrose};
  \node[font=\scriptsize, right] at ($(cohom)!0.5!(holo)$) {regularization};
  \node[font=\scriptsize, right] at ($(holo)!0.5!(rho)$) {Definition 5.1};
  \node[font=\scriptsize, below] at ($(rho)!0.5!(E)$) {consistency};
  \node[font=\scriptsize, below] at ($(E)!0.5!(dyn)$) {Theorem 6.2};
  
\end{tikzpicture}
\caption{Conceptual flow from conformal structure to emergent dynamics. The conformal density $\rho$ arises as an intrinsic measure of holomorphic content, and its consistency conditions yield dynamical equations without introducing external structure.}
\label{fig:conceptual-flow}
\end{figure}


%% =============================================================================
%% FIGURA 3: REGULARIZATION SCHEME
%% =============================================================================
%% Ubicación sugerida: Sección 5.2

\begin{figure}[htbp]
\centering
\begin{tikzpicture}[scale=1.2]
  % Axes
  \draw[-{Stealth}] (-0.5,0) -- (5,0) node[right] {$\Lambda$};
  \draw[-{Stealth}] (0,-0.5) -- (0,3.5) node[above] {$\mathrm{Tr}(\chi_\Lambda P)$};
  
  % Horizontal asymptote (rank P)
  \draw[dashed, gray] (0,3) -- (5,3);
  \node[left] at (0,3) {$N$};
  
  % Curve converging to N
  \draw[thick, blue!70!black, domain=0.3:4.8, samples=100] 
    plot (\x, {3*(1 - exp(-0.8*\x))});
  
  % Points
  \foreach \x/\y in {1/1.65, 2/2.52, 3/2.88, 4/2.97} {
    \fill[blue!70!black] (\x, \y) circle (2pt);
  }
  
  % Labels
  \node[below] at (1,0) {$\Lambda_1$};
  \node[below] at (2,0) {$\Lambda_2$};
  \node[below] at (3,0) {$\Lambda_3$};
  \node[below] at (4,0) {$\Lambda_4$};
  
  % Annotation
  \draw[-{Stealth}, thick] (3.5,1.5) -- (4.3,2.8);
  \node[align=center, font=\small] at (3,1) {Convergence\\$\Lambda \to \infty$};
  
  % Dimension label
  \node[right, font=\small] at (5,3) {$= \dim H^1 = n+1$};
  
\end{tikzpicture}
\caption{Regularized trace convergence (Lemma~5.3). As the cutoff parameter $\Lambda \to \infty$, the regularized trace $\mathrm{Tr}(\chi_\Lambda P)$ converges monotonically to the algebraic dimension $N = n+1$ of the cohomology space.}
\label{fig:trace-convergence}
\end{figure}


%% =============================================================================
%% FIGURA 4: VARIABLE RHO PROFILES (SCHEMATIC)
%% =============================================================================
%% Ubicación sugerida: Appendix D o Section 5

\begin{figure}[htbp]
\centering
\begin{tikzpicture}[scale=1.0]
  % Axes
  \draw[-{Stealth}] (-0.5,0) -- (6,0) node[right] {$r$};
  \draw[-{Stealth}] (0,-0.3) -- (0,3) node[above] {$\rho(r)$};
  
  % Horizontal line at 1
  \draw[dashed, gray] (0,2) -- (6,2);
  \node[left] at (0,2) {$1$};
  
  % Vacuum (constant)
  \draw[thick, blue, domain=0:5.5] plot (\x, 2);
  \node[blue, right] at (5.5,2.2) {vacuum};
  
  % Radial decay
  \draw[thick, red, domain=0:5.5, samples=100] 
    plot (\x, {2/(1 + 0.3*\x*\x)});
  \node[red, right] at (5.5,0.5) {radial};
  
  % Compact support
  \draw[thick, green!50!black, domain=0:2] 
    plot (\x, {0.8 + 1.2*\x*\x/4});
  \draw[thick, green!50!black, domain=2:5.5] plot (\x, 2);
  \node[green!50!black, right] at (5.5,1.3) {compact};
  
  % Mark R
  \draw[dotted] (2,0) -- (2,2);
  \node[below] at (2,0) {$R$};
  
  % Labels
  \node[left] at (0,0.8) {$\rho_{\min}$};
  \draw[dotted, gray] (0,0.8) -- (0.1,0.8);
  
\end{tikzpicture}
\caption{Schematic profiles of the conformal density $\rho(r)$. Blue: Minkowski vacuum ($\rho \equiv 1$). Red: radial variation $\rho = (1 + r^2/R^2)^{-1}$. Green: compactly supported perturbation with $\rho = 1$ for $r > R$.}
\label{fig:rho-profiles}
\end{figure}


%% =============================================================================
%% TABLA 1: COMPARISON OF FREQUENCY CHARACTERIZATIONS
%% =============================================================================

\begin{table}[htbp]
\caption{Comparison of positive frequency characterizations in field theory}
\label{tab:frequency-comparison}
\centering
\begin{tabular}{lccc}
\hline\hline
\textbf{Approach} & \textbf{Conformal inv.} & \textbf{Global} & \textbf{Dynamical DOF} \\
\hline
Fourier decomposition & No & No & No \\
Forward tube (BHW) & Partial & No & No \\
Twistor space (Penrose) & Yes & Yes & No \\
\textbf{Conformal density (this work)} & \textbf{Yes} & \textbf{Yes} & \textbf{Yes} \\
\hline\hline
\end{tabular}
\end{table}


%% =============================================================================
%% TABLA 2: PROPERTIES OF THE INTEGRABILITY OPERATOR
%% =============================================================================

\begin{table}[htbp]
\caption{Analytic properties of the integrability operator $\mathcal{E}$}
\label{tab:E-properties}
\centering
\begin{tabular}{ll}
\hline\hline
\textbf{Property} & \textbf{Description} \\
\hline
Order & Second-order, quasi-linear \\
Principal symbol & $\sigma_{\mathcal{E}}(\xi) = \alpha|\xi|^2 + \beta P^{\mu\nu}\xi_\mu\xi_\nu + O(|W|)$ \\
Ellipticity & Uniform for $\alpha > 0$, conformally flat backgrounds \\
Conformal covariance & $\mathcal{E}_{\Omega^2 g}[\rho] = \Omega^{-w}\mathcal{E}_g[\rho]$ \\
Variational structure & $\mathcal{E}[\rho] = \delta S / \delta\rho$ (Helmholtz conditions satisfied) \\
\hline\hline
\end{tabular}
\end{table}


%% =============================================================================
%% TABLA 3: COMPARISON WITH EMERGENT GRAVITY PROGRAMS
%% =============================================================================

\begin{table}[htbp]
\caption{Comparison with emergent gravity programs}
\label{tab:emergent-comparison}
\centering
\begin{tabular}{lcccc}
\hline\hline
\textbf{Program} & \textbf{Primary ontology} & \textbf{Conformal} & \textbf{No metric input} & \textbf{Dynamical} \\
\hline
Jacobson (thermodynamic) & Entropy & Partial & No & Yes \\
Verlinde (entropic) & Information & No & No & Yes \\
LQG (spin networks) & Discrete geometry & No & Yes & Yes \\
Causal sets & Discrete causality & No & Yes & Partial \\
\textbf{This work} & \textbf{Holomorphic structure} & \textbf{Yes} & \textbf{Yes} & \textbf{Yes} \\
\hline\hline
\end{tabular}
\end{table}
